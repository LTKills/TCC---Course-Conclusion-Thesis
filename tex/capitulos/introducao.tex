\section{Motivação e Contextualização}
A Internet das Coisas (em inglês, Internet of Things ou IoT) é um conceito em grande expansão atualmente (TODO: REFERÊNCIA). As máquinas de IoT são em sua grande maioria sensores, microcontroladores e sistemas embarcados \-- em suma, sistemas computacionais pequenos, que gastam pouca energia e possuem baixo poder computacional. 

A popularização da Internet das Coisas ocorreu tendo como principais focos o baixo consumo energético, leveza computacional (TODO: LEVEZA?), e solução de diversos problemas de viabilidade em sistemas críticos como, por exemplo, máquinas que precisam operar em ambientes de altas temperatura e pressão (TODO: REFERÊNCIA).

Esse desenvolvimento acelerado, no entanto, deixou de lado um aspecto importante de qualquer sistema computacional: segurança. Logo, percebeu-se uma gama de falhas de segurança em divesos protocolos de comunicação (TODO: REFERÊNCIA) projetados para sistemas IoT, bem como a falta de preocupação com criptografia e outras práticas de segurança.

Essa crise de segurança despertou uma enorme necessidade de protocolos seguros, como aqueles mais tradicionais (TODO: REFERÊNCIA) \-- utilizados há anos fora da Internet das Coisas \-- mas que ainda possuíssem as propriedades de baixos custos energético e computacional, intrínsecas à IoT.


\subsection{Modelo Publish/Subscribe}
A Internet das Coisas, como o próprio nome dá a entender, faz parte da Internet, ou seja, utiliza os protocolos TCP/IP para comunicação (TODO: REFERÊNCIA), no entanto, protocolos como HTTP e HTTPS, largamente utilizado nas demais camadas da rede, não são adequados para a maioria das aplicações IoT devido a restrições como baixa largura de banda e alta latência (TODO: REFERÊNCIA). O modelo Publish/Subscribe é um dos modelos de comunicação mais utilizados por dispositivos IoT, sobretudo em redes de sensores (TODO: REFERÊNCIA).

Nesse modelo os clientes, máquinas que se comunicam, se conectam a um corretor, ou broker em inglês, e se inscrevem (TODO: subscribe) em tópicos. Os clientes podem ainda publicar (TODO: publish) mensagens em um ou mais tópicos, e todos os clientes que estiverem inscritos nesses tópicos receberão as mensagens.

TODO: IBAGENS!

Esse modelo permite que apenas o broker despenda uma quantidade considerável de recursos energéticos e computacionais \-- para armazenar as listas dos clientes inscritos, garantir níveis de qualidade de serviço, dentre outras tarefas, \-- enquanto os clientes somente enviam e recebem as mensagens, não precisando nem mesmo checar os tópicos ativamente (TODO: MELHORAR, REFERÊNCIA, BUSY WAIT).


\subsection{MQTT}
O MQTT, Message Queuing Telemetry Transfer, é um protocolo publish/subscribe criado em 1999 por Andy Stanford-Clark e Arlen Nipper para [], e é o que usaremos para estudo de caso

-qos
-achou mercado agora com iot


\subsection{DoS}
- sigla
- histórico
- iot
- mqtt (publish/subscribe



\section{Objetivos}



\section{Organização}



% Comando simples para exibir comandos Latex no texto
\newcommand{\comando}[1]{\textbf{$\backslash$#1}}

Este documento explica brevemente como trabalhar com a classe \LaTeX~\textit{icmc} para confeccionar trabalhos acadêmicos seguindo as normas da \sigla{ABNT}{Associação Brasileira de Normas Técnicas} e as \aspas{\textit{Diretrizes para apresentação de dissertações e teses da USP: documento eletrônico e impresso. Parte I (ABNT)}}, publicado pelo \sigla{SIBi}{Sistema Integrado de Bibliotecas} USP. O presente manual também atende as exigências prevista no regimento do Programa de Pós-graduação em \sigla{CCMC}{Ciências da Computação e Matemática Computacional} do \sigla{ICMC}{Instituto de Ciências Matemáticas e de Computação} da \sigla{USP}{Universidade de São Paulo}.


A classe \textit{icmc} foi construída com base na última versão da classe \textit{abntex2} e do pacote \textit{abntex2cite}. Portanto, este documento exemplifica a elaboração de trabalho
acadêmico (tese, dissertação e outros do gênero) produzido conforme a ABNT NBR
14724:2011 \textit{Informação e documentação - Trabalhos acadêmicos - Apresentação}.

Assim, é altamente recomendável que seja consultada a documentação do \textit{abntex2}\footnote{https://code.google.com/p/abntex2/downloads/list}. A classe \textit{abntex2} foi desenvolvida para facilitar a escrita de documentos seguindo as normas da ABNT no ambiente \LaTeX\;\cite{frasson:2005:classe_abnt}.

Todo o trabalho de pesquisa e ajustes da presente classe \LaTeX~\emph{icmc} foram feitos pelo aluno mestrado do Programa de Pós-graduação em Ciência da Computação e Matemática Computacional, Humberto Lidio Antonelli, durante a confecção da sua monografia de qualificação.

O requisito básico para utilização da classe \textit{icmc} é criar um documento desta classe com o comando
\comando{documentclass[@parameters]\{icmc\}} e ter, no diretório de trabalho, o arquivo \emph{icmc.cls} presente. Entretanto, recomenda-se fortemente manter a estrutura de diretório inicial fornecida por este modelo.

Os parâmetros possíveis utilizados pelo \comando{documentclass} são:
\begin{description}
\item[french, spanish, english, brazil] Adiciona o idioma para correta hifenização correta no documento. O último idioma declarado é o principal do documento. O valor padrão é \textbf{brazil}.
\end{description}

