\section{Motivação e Contextualização}
A Internet das Coisas (em inglês, Internet of Things ou IoT) é um conceito em grande expansão atualmente (TODO: REFERÊNCIA). As máquinas de IoT são em sua grande maioria sensores, microcontroladores e sistemas embarcados \-- em suma, sistemas computacionais pequenos, que gastam pouca energia e possuem baixo poder computacional. 

A popularização da Internet das Coisas ocorreu tendo como principais focos o baixo consumo energético, leveza computacional (TODO: LEVEZA?), e solução de diversos problemas de viabilidade em sistemas críticos como, por exemplo, máquinas que precisam operar em ambientes de altas temperatura e pressão (TODO: REFERÊNCIA).

Esse desenvolvimento acelerado, no entanto, deixou de lado um aspecto importante de qualquer sistema computacional: segurança. Logo, percebeu-se uma gama de falhas de segurança em diversos protocolos de comunicação (TODO: REFERÊNCIA) projetados para sistemas IoT, bem como a falta de preocupação com criptografia e outras práticas de segurança.

Tamanha era a magnitude essa crise, logo viu-se uma enorme necessidade de módulos e adendos de segurança nesses protocolos, tal como há para protocolos mais tradicionais (TODO: REFERÊNCIA) \-- utilizados há anos fora do contexto de Internet das Coisas \-- mas, ainda, com o desafio de que possuíssem as propriedades de baixos custos energético e computacional, tão importantes para os dispositivos de IoT.


\subsection{Protocolos em IoT e Modelo Publish/Subscribe}
A Internet das Coisas, como o próprio nome dá a entender, faz parte da Internet, ou seja, utiliza, em geral, os protocolos TCP/IP para comunicação (TODO: REFERÊNCIA). Se por um lado TCP/IP são comuns mesmo no contexto de IoT, por outro, protocolos como HTTP e HTTPS, largamente utilizados nas demais camada Web da rede, não o são.

Dessa forma, os protocolos utilizados para troca de dados na Web não se mostram adequados para a maioria das aplicações IoT, justamente devido a restrições como pouca largura de banda, latência, dentre outros (TODO: REFERÊNCIA).

O modelo Publish/Subscribe é um dos modelos de comunicação mais largamente utilizados por dispositivos IoT, sobretudo em redes de sensores (TODO: REFERÊNCIA). Nesse modelo os clientes, máquinas que se comunicam, se conectam a um corretor, ou broker em inglês, e se inscrevem (TODO: subscribe) em tópicos. Os clientes podem ainda publicar (TODO: publish) mensagens em um ou mais tópicos, e todos os clientes que estiverem inscritos nesses tópicos receberão as mensagens.

TODO: IBAGENS!

Esse modelo permite que apenas o broker despenda uma quantidade considerável de recursos energéticos e computacionais \-- para armazenar as listas dos clientes inscritos, garantir níveis de qualidade de serviço, dentre outras tarefas, \-- enquanto os clientes somente enviam e recebem as mensagens, não precisando nem mesmo checar os tópicos ativamente (TODO: MELHORAR, REFERÊNCIA, BUSY WAIT).


\subsection{MQTT}
Criado em 1999 por Andy Stanford-Clark e Arlen Nipper, o MQTT, Message Queuing Telemetry Transfer, é um protocolo publish/subscribe  que tem como principal objetivo baixa utilização de banda da rede. O MQTT encontrou um mercado promissor ainda recentemente, com a popularização da Internet das Coisas.

- qos
- control packet types (mqtt header)





\subsection{DoS}
Ataques de negação de serviço, em inglês Denial of Service, buscam desestabilizar um sistema de maneira que os clientes legítimos não consigam acessá-lo. Ataques DoS frequentemente utilizam ferramentas que realizam uma série requisições repetidas para o servidor alvo com vistas a ocupar canais de clientes legítimos e, assim, comprometer a disponibilidade do sistema.

Apesar desse tipo de vulnerabilidade já ter sido explorada e estudada há anos, recentemente, por conta da popularização da Internet das Coisas, novos problemas relacionados a ela surgiram, sobretudo no que tange a IoT.


\subsection{Mosquitto}
O protocolo MQTT, assim como qualquer protocolo em redes de computadores, não é senão um conjunto de especificações padronizadas para comunicação de máquinas. Sendo assim, existem várias diferentes implementações do protocolo MQTT, cada qual com suas particularidades em relação a modos de armazenamento, algoritmos de decisão etc.

O Mosquitto (TODO: REFERÊNCIA) é um projeto da organização Eclipse (TODO: REFERÊNCIA) que visa a implementar os componentes especificados no padrão do protocolo MQTT, dentre eles, destacam-se o corretor, Mosquitto broker, e o cliente, Paho.

- como funciona??








\section{Objetivos}

- ataques DoS
- implementar ataques
- proof of concept de algumas ideias
- vulnerabilidades antigas mas (possivelmente) ainda presentes
- TCP SYN flooding
- SSL apresenta grande overhead, por isso não vamos focar em ataques que exploram sistemas baseados em SSL



\section{Organização}


