% ------------------------------------------------------------------------
% ------------------------------------------------------------------------
% ICMC: Modelo de Trabalho Acadêmico (tese de doutorado, dissertação de
% mestrado e trabalhos monográficos em geral) em conformidade com 
% ABNT NBR 14724:2011: Informação e documentação - Trabalhos acadêmicos -
% Apresentação
% ------------------------------------------------------------------------
% ------------------------------------------------------------------------

% Opções: 
%   Qualificação         = qualificacao 
%   Curso                = doutorado/mestrado
%   Situação do trabalho = pre-defesa/pos-defesa (exceto para qualificação)
% -- opções do pacote babel --
% Idioma padrão = brazil
	%french,	    	% idioma adicional para hifenização
	%spanish,			% idioma adicional para hifenização
	%english,			% idioma adicional para hifenização
	%brazil				% o último idioma é o principal do documento
\documentclass[brazil]{packages/icmc}

% ---
% Pacotes Opcionais
% ---
\usepackage{rotating}           % Usado para rotacionar o texto
\usepackage[all,knot,arc,import,poly]{xy}   % Pacote para desenhos gráficos
% Este pacote pode conflitar com outros pacotes gráficos como o ``pictex''
% Então é necessário usar apenas um dos pacotes conflitantes


% ---
% Informações de dados para CAPA e FOLHA DE ROSTO
% ---
\titulo{Estudo de ataques DoS em redes MQTT}
\autor[Cruz, G. M.]{Gabriel de Melo Cruz}
\orientador[Orientadora:]{Profa. Dra.}{Kalinka Regina Lucas Jaquie Castelo Branco}
%\coorientador[Coorientadora:]{Doutoranda}{Mariana Rodrigues}
\curso{Computação}
\area{Segurança em Redes de Computadores} % Área de concentração do trabalho
\data{12}{06}{2019} % Data do depósito
% ---

% ---
% RESUMOS
% ---

% Resumo em português
% conter no máximo 500 palavras
\textoresumo{
    O desenvolvimento de tecnologias associadas à Internet das Coisas está hoje mais rápido do que nunca. Diante desse cenário de evolução desenfreada, simples práticas de segurança frequentemente não são contempladas, deixando esses sistemas expostos através de várias vulnerabilidades. Neste trabalho exploramos falhas que permitem desabilitar o serviço em um dos protocolos mais utilizados por sistemas IoT: o MQTT. A primeira vulnerabilidade explorada leva em conta a utilização do protocolo TCP -- que oferece ao MQTT o serviço de conexão e entrega de mensagens -- incapacitando a criação de novas conexões. Ainda, exploramos o limite de banda do servidor, ou broker, no protocolo MQTT, fazendo com que ele ocupe-a unicamente com as requisições do atacante. Os dois ataques aqui implementados, apesar de relativamente simples, são extremamente eficazes sobretudo devido à falta de implementações de software seguras em sistemas IoT.
    }{IoT, Internet das Coisas, MQTT, Segurança, Ataque, Vulnerabilidade}

% ---
% resumo em inglês
% ---
\textoresumo[english]{
    The development of technologies associated with the Internet of Things is faster than ever before. Facing this scenario of unbridled evolution, simple security practices are often not contemplated, leaving these systems exposed through numerous vulnerabilities. In this work we explore failures that allow us disable the service in one of the most used protocols on IoT systems: MQTT. The first exploited vulnerability makes use of a problem in the TCP protocol -- which provides MQTT with the connection and delivery of messages --, allowing us to prevent new connections on the server, or broker. We also exploit issues regarding the bandwidth limit of the broker, occupying it exclusively with the attacker's requests. Even though the two attacks implemented here are relatively simple, they are also extremely effective, mainly because of the lack of secure software implementations in IoT systems.
    }{IoT, Internet of Things, MQTT, Security, Attack, Vulnerability}
    
% ---
% Configurações de aparência do PDF final
% ---
% alterando o aspecto da cor azul
\definecolor{blue}{RGB}{41,5,195}

% informações do PDF
\makeatletter
\hypersetup{
     	pagebackref=true,
		pdftitle={\@title}, 
		pdfauthor={\@author},
    	pdfsubject={\imprimirpreambulo},
	    pdfcreator={LaTeX with abnTeX2/ICMC-USP},
		pdfkeywords={\palavraschave}, 
		colorlinks=true,       		% false: boxed links; true: colored links
    	linkcolor=blue,          	% color of internal links
    	citecolor=blue,        		% color of links to bibliography
    	filecolor=magenta,      	% color of file links
		urlcolor=blue,
		bookmarksdepth=4
}
\makeatother
% --- 

% ----------------------------------------------------------
% ELEMENTOS PRÉ-TEXTUAIS
% ----------------------------------------------------------

% Inserir a ficha catalográfica
%\incluifichacatalografica[tex/fichaCatalografica.pdf]
%\incluifichacatalografica


% Inserir folha de aprovação
%
% Isto é um exemplo de Folha de aprovação, elemento obrigatório da NBR
% 14724/2011 (seção 4.2.1.3). Você pode utilizar este modelo até a aprovação
% do trabalho. Após isso, substitua todo o conteúdo deste arquivo por uma
% imagem da página assinada pela banca com o comando abaixo:
%
% \includepdf{folhadeaprovacao_final.pdf}
%
\begin{folhadeaprovacao}

  \begin{center}
    {\ABNTEXchapterfont\large\imprimirautor}

    \vspace*{\fill}\vspace*{\fill}
    {\ABNTEXchapterfont\bfseries\Large\imprimirtitulo}
    \vspace*{\fill}
    
    \hspace{.45\textwidth}
    \begin{minipage}{.5\textwidth}
        \imprimirpreambulo
    \end{minipage}%
    \vspace*{\fill}
   \end{center}
    
   Trabalho aprovado. \imprimirlocal, 24 de novembro de 2012:

   \assinatura{\textbf{\imprimirorientador} \\ Orientador} 
   \assinatura{\textbf{Professor} \\ Convidado 1}
   \assinatura{\textbf{Professor} \\ Convidado 2}
   %\assinatura{\textbf{Professor} \\ Convidado 3}
   %\assinatura{\textbf{Professor} \\ Convidado 4}
      
   \begin{center}
    \vspace*{0.5cm}
    {\large\imprimirlocal}
    \par
    {\large\imprimirdata}
    \vspace*{1cm}
  \end{center}
  
\end{folhadeaprovacao}
% ---

% DEDICATÓRIA / AGRADECIMENTO / EPÍGRAFE
\textodedicatoria*{tex/pre-textual/dedicatoria}
% \textoagradecimentos*{tex/pre-textual/agradecimentos}
\textoepigrafe*{tex/pre-textual/epigrafe}

% Inclui a lista de figuras
\incluilistadefiguras

% Inclui a lista de tabelas
%\incluilistadetabelas

% Inclui a lista de quadros
%\incluilistadequadros

% Inclui a lista de algoritmos
%\incluilistadealgoritmos

% Inclui a lista de códigos
\incluilistadecodigos

% Inclui a lista de siglas e abreviaturas
\incluilistadesiglas

% Inclui a lista de símbolos
%\incluilistadesimbolos

% ----
% Início do documento
% ----
\begin{document}

% ----------------------------------------------------------
% ELEMENTOS TEXTUAIS
% ----------------------------------------------------------
\textual

\chapter{Introdução}
\label{chapter:introducao}
\section{Motivação e Contextualização}
A Internet das Coisas (em inglês, Internet of Things ou IoT) é um conceito em grande expansão atualmente (TODO: REFERÊNCIA). As máquinas de IoT são em sua grande maioria sensores, microcontroladores e sistemas embarcados \-- em suma, sistemas computacionais pequenos, que gastam pouca energia e possuem baixo poder computacional. 

A popularização da Internet das Coisas ocorreu tendo como principais focos o baixo consumo energético, leveza computacional (TODO: LEVEZA?), e solução de diversos problemas de viabilidade em sistemas críticos como, por exemplo, máquinas que precisam operar em ambientes de altas temperatura e pressão (TODO: REFERÊNCIA).

Esse desenvolvimento acelerado, no entanto, deixou de lado um aspecto importante de qualquer sistema computacional: segurança. Logo, percebeu-se uma gama de falhas de segurança em diversos protocolos de comunicação (TODO: REFERÊNCIA) projetados para sistemas IoT, bem como a falta de preocupação com criptografia e outras práticas de segurança.

Tamanha era a magnitude essa crise, logo viu-se uma enorme necessidade de módulos e adendos de segurança nesses protocolos, tal como há para protocolos mais tradicionais (TODO: REFERÊNCIA) \-- utilizados há anos fora do contexto de Internet das Coisas \-- mas, ainda, com o desafio de que possuíssem as propriedades de baixos custos energético e computacional, tão importantes para os dispositivos de IoT.


\subsection{Protocolos em IoT e Modelo Publish/Subscribe}
A Internet das Coisas, como o próprio nome dá a entender, faz parte da Internet, ou seja, utiliza, em geral, os protocolos TCP/IP para comunicação (TODO: REFERÊNCIA). Se por um lado TCP/IP são comuns mesmo no contexto de IoT, por outro, protocolos como HTTP e HTTPS, largamente utilizados nas demais camada Web da rede, não o são.

Dessa forma, os protocolos utilizados para troca de dados na Web não se mostram adequados para a maioria das aplicações IoT, justamente devido a restrições como pouca largura de banda, latência, dentre outros (TODO: REFERÊNCIA).

O modelo Publish/Subscribe é um dos modelos de comunicação mais largamente utilizados por dispositivos IoT, sobretudo em redes de sensores (TODO: REFERÊNCIA). Nesse modelo os clientes, máquinas que se comunicam, se conectam a um corretor, ou broker em inglês, e se inscrevem (TODO: subscribe) em tópicos. Os clientes podem ainda publicar (TODO: publish) mensagens em um ou mais tópicos, e todos os clientes que estiverem inscritos nesses tópicos receberão as mensagens.

TODO: IBAGENS!

Esse modelo permite que apenas o broker despenda uma quantidade considerável de recursos energéticos e computacionais \-- para armazenar as listas dos clientes inscritos, garantir níveis de qualidade de serviço, dentre outras tarefas, \-- enquanto os clientes somente enviam e recebem as mensagens, não precisando nem mesmo checar os tópicos ativamente (TODO: MELHORAR, REFERÊNCIA, BUSY WAIT).


\subsection{MQTT}
Criado em 1999 por Andy Stanford-Clark e Arlen Nipper, o MQTT, Message Queuing Telemetry Transfer, é um protocolo publish/subscribe  que tem como principal objetivo baixa utilização de banda da rede. O MQTT encontrou um mercado promissor ainda recentemente, com a popularização da Internet das Coisas.

- qos
- control packet types (mqtt header)





\subsection{DoS}
Ataques de negação de serviço, em inglês Denial of Service, buscam desestabilizar um sistema de maneira que os clientes legítimos não consigam acessá-lo. Ataques DoS frequentemente utilizam ferramentas que realizam uma série requisições repetidas para o servidor alvo com vistas a ocupar canais de clientes legítimos e, assim, comprometer a disponibilidade do sistema.

Apesar desse tipo de vulnerabilidade já ter sido explorada e estudada há anos, recentemente, por conta da popularização da Internet das Coisas, novos problemas relacionados a ela surgiram, sobretudo no que tange a IoT.


\subsection{Mosquitto}
O protocolo MQTT, assim como qualquer protocolo em redes de computadores, não é senão um conjunto de especificações padronizadas para comunicação de máquinas. Sendo assim, existem várias diferentes implementações do protocolo MQTT, cada qual com suas particularidades em relação a modos de armazenamento, algoritmos de decisão etc.

O Mosquitto (TODO: REFERÊNCIA) é um projeto da organização Eclipse (TODO: REFERÊNCIA) que visa a implementar os componentes especificados no padrão do protocolo MQTT, dentre eles, destacam-se o corretor, Mosquitto broker, e o cliente, Paho.

- como funciona??








\section{Objetivos}

- ataques DoS
- implementar ataques
- proof of concept de algumas ideias
- vulnerabilidades antigas mas (possivelmente) ainda presentes
- TCP SYN flooding
- SSL apresenta grande overhead, por isso não vamos focar em ataques que exploram sistemas baseados em SSL



\section{Organização}




\chapter{Métodos, Técnicas e Tecnologias Utilizadas}
\label{chapter:metodos}




\chapter{Desenvolvimento}
\label{chapter:desenvolvimento}
\section{O Problema}

\section{Atividades Realizadas}

\section{Resultados}

\section{Dificuldades e Limitações}



\chapter{Conclusão}
\label{chapter:conclusao}

\section{Resultados}



\section{Dificuldades e Limitações}



%\chapter{Citações e Referências}
%\label{chapter:citacoes}
%Em documentos acadêmicos podem existir citações diretas e citações indiretas. As citações indiretas são feitas quando se reescreve uma referência consultada. Nas citações indiretas há duas formatações possíveis dependendo de como ocorre a citação no texto. Quando o autor é mencionado explicitamente  deve ser usado o comando \comando{citeonline\{\}}, nas demais situações é usado o comando \comando{cite\{\}}. No quadro \ref{figura:citacao_indireta_explicita} encontrasse um  exemplo de uso do comando \comando{citeonline\{\}}.

\begin{quadro}[htb]
\caption{Exemplo de citação indireta explícita} \label{figura:citacao_indireta_explicita}
\hrulefill

\lstset{language=Tex, breaklines=true}
\begin{lstlisting}
Segundo \citeonline{silveira:2006}, o trabalho de conclusão de curso deve seguir as normas da ABNT.
\end{lstlisting}

\hrulefill

Segundo \citeonline{silveira:2006:manual_tcc}, o trabalho de conclusão de curso deve seguir as normas da ABNT.

\hrulefill

%\legend{Fonte: o autor.}
\end{quadro}

Para especificar a página consultada na referência é preciso acrescentá-la entre colchetes com os comandos \comando{cite[página]\{\}} ou \comando{citeonline[página]\{\}}. No quadro \ref{figura:citacao_indireta_pagina} é mostrado um exemplo de citação com página específica.

\begin{quadro}[htb]
\caption{Exemplo de citação indireta não explícita} \label{figura:citacao_indireta_pagina}
\hrulefill

\lstset{language=Tex, breaklines=true}
\begin{lstlisting}
A folha de aprovação é um elemento obrigatório na monografia de projeto final de curso trabalho de conclusão de curso.  \cite[p.~10]{silveira:2006}.
\end{lstlisting}

\hrulefill

A folha de aprovação é um elemento obrigatório no trabalho de conclusão de curso.  \cite[p.~10]{silveira:2006:manual_tcc}.

\hrulefill

\end{quadro}

As citações diretas acontecem quando o texto de uma referência é transcrito literalmente. As citações diretas são curtas (até três linhas) são inseridas no texto entre aspas duplas. Conforme exemplo no quadro \ref{figura:citacao_direta_curta}.

\begin{quadro}[htb]
\caption{Exemplo de citação direta curta}
\label{figura:citacao_direta_curta}
\hrulefill

\lstset{language=Tex, breaklines=true}
\begin{lstlisting}
``Os quadros, ao contrário das tabelas, apresentam dados textuais e devem localizar-se o mais próximo do texto a que se referem'' \cite[p.~25]{silveira:2006}.
\end{lstlisting}

\hrulefill

``Os quadros, ao contrário das tabelas, apresentam dados textuais e devem localizar-se o mais próximo do texto a que se referem'' \cite[p.~25]{silveira:2006:manual_tcc}.

\hrulefill
\end{quadro}

As citações longas (com mais de 3 linhas) podem ser inseridas via \comando{begin\{citacao\}} conforme quadro \ref{figura:citacao_direta_longa}.

\begin{quadro}[htb]
\caption{Exemplo de citação direta longa}
\label{figura:citacao_direta_longa}
\hrulefill

\lstset{language=Tex, breaklines=true}
\begin{lstlisting}
\begin{citacao}
Síntese final do trabalho, a conclusão constitui-se de uma resposta à hipótese enunciada na introdução. O autor manifestará seu ponto de vista sobre os resultados obtidos e sobre o alcance dos mesmos. Não se permite a inclusão de dados novos nesse capítulo nem citações ou interpretações de outros autores \cite[p.~25]{silveira:2006}.
\end{citacao}
\end{lstlisting}

\hrulefill

\begin{citacao}
Síntese final do trabalho, a conclusão constitui-se de uma resposta à hipótese enunciada na introdução. O autor manifestará seu ponto de vista sobre os resultados obtidos e sobre o alcance dos mesmos. Não se permite a inclusão de dados novos nesse capítulo nem citações ou interpretações de outros autores \cite[p.~25]{silveira:2006:manual_tcc}.
\end{citacao}

\hrulefill

\end{quadro}


% ---
% Finaliza a parte no bookmark do PDF, para que se inicie o bookmark na raiz
% ---
\bookmarksetup{startatroot}% 
% ---

% ----------------------------------------------------------
% ELEMENTOS PÓS-TEXTUAIS
% ----------------------------------------------------------
\postextual

% ----------------------------------------------------------
% Referências bibliográficas
% ----------------------------------------------------------
\bibliography{references}

% ---------------------------------------------------------------------
% GLOSSÁRIO
% ---------------------------------------------------------------------

% Arquivo que contém as definições que vão aparecer no glossário
%\newword{1}{Agentes de usuário}{qualquer \textit{software} que recupera, processa e facilita a interação do usuário final com o conteúdo  Web. Como exemplo desses agentes, podem ser citados navegadores, reprodutores multimídia e tecnologias assistivas} 

\newword{2}{ATAG}{\textit{Authoring Tool Accessibility Guidelines} ou Diretrizes de acessibilidade para ferramentas de autoria. É um conjunto de diretrizes para desenvolvedores de qualquer ferramenta de criação para Web, como: simples editores HTML, ferramentas para exportar conteúdo para Web, ferramentas multimídia e sistemas de gerenciamento de conteúdo}

\newword{3}{DOM}{\textit{Document Object Model} ou Modelo de Objetos de Documentos. É uma especificação do W3C para organizar objetos de um documento em que se pode, dinamicamente, alterar e editar sua estrutura, conteúdo e estilo}

\newword{4}{Javascript}{é uma linguagem de \textit{script} para desenvolvimento de certos tratamentos 
que ocorrem lado do cliente, geralmente o navegador Web. Ela é utilizada geralmente quando 
é inconveniente ou impossível para o servidor para fazer esse tratamento}

\newword{5}{Stakeholder}{qualquer pessoa ou grupo, que legitima as ações de uma organização. É formado pelos funcionários da empresa, gestores, gerentes, proprietários, fornecedores, clientes, o Estado, credores, sindicatos e diversas outras pessoas ou empresas que estejam relacionadas com uma determinada ação ou projeto.}

\newword{6}{Tag}{ou etiqueta, é uma palavra-chave ou termo associado com uma informação que a descreve. Em linguagens de marcação, como o HTML, consistem em breves instruções, tendo uma marca de início e outra de fim para que o navegador possa mostrar a renderização da página}

\newword{7}{Tecnologia assistiva}{Conjunto de técnicas, aparelhos, instrumentos, produtos e procedimentos que visam auxiliar a mobilidade, percepção e utilização do meio ambiente e dos elementos por pessoas com deficiência	}

\newword{8}{UAAG}{\textit{User Agent Accessibility Guidelines} ou Diretrizes de acessibilidade para agentes de usuário. Conjuntos de diretrizes para desenvolvedores de agentes de usuário (por exemplo: navegadores e reprodutores de mídia) com a finalidade de fazer com que tais agentes permitam sua utilização adequada por pessoas com algum tipo de deficiência}

\newword{9}{W3C}{\textit{World Wide Web Consortium}. É um consórcio internacional formado por empresas, órgãos governamentais e organizações independentes que visa desenvolver padrões para a criação e a interpretação de conteúdos da Web}

\newword{10}{WAI}{\textit{Web Accessibility Initiative} ou Iniciativa de Acessibilidade na Web. É a iniciativa do W3C no que tange a desenvolver estratégias, diretrizes e outros recursos, a fim de que as informações presentes na Web sejam acessíveis para pessoas com ou sem deficiência}

\newword{11}{WCAG}{\textit{Web Content Accessibility Guidelines} ou Diretrizes de Acessibilidade para Conteúdo Web. É um conjunto de diretrizes criado pelo W3C para auxílio na elaboração de conteúdo acessível, que atualmente está em sua versão 2.0 desde 2008.}

\newword{12}{WYSIWYG}{``What You See Is What You Get''  ou ``O que você vê é o que você obtém''.  Recurso tem por objetivo permitir que um documento, enquanto manipulado na tela, tenha a mesma aparência de sua utilização, usualmente sendo considerada final. Isso facilita para o desenvolvedor que pode trabalhar visualizando a aparência do documento sem precisar salvar em vários momentos e abrir em um \textit{software} separado de visualização}

\newword{13}{Desenho universal}{concepção de produtos, ambientes, programas e serviços a serem usados, na maior medida possível, por todas as pessoas, sem necessidade de adaptação ou projeto específico. O desenho universal não excluirá as ajudas técnicas para grupos específicos de pessoas com deficiência, quando necessárias}

\newword{14}{Mobilidade reduzida}{dificuldade permanente ou temporária que uma pessoa tem para se movimentar, gerando redução efetiva da mobilidade, flexibilidade, coordenação motora e percepção}

\newword{15}{Pessoas com deficiência}{aquelas que têm impedimentos de longo prazo de natureza física, mental, intelectual ou sensorial, os quais, em interação com diversas barreiras, podem obstruir sua participação plena e efetiva na sociedade em igualdades de condições com as demais pessoas. Atualmente chegou-se a um consenso quanto à utilização da expressão ``pessoa com deficiência'' em todas as suas manifestações orais ou escritas, em lugar de termos como ``deficiente'', ``pessoa portadora de deficiência'', ``pessoa com necessidades especiais'' e ``portador de necessidades especiais''}

\newword{16}{Framework}{é uma abstração que une códigos comuns entre vários projetos de \textit{software} provendo uma funcionalidade genérica. \textit{Frameworks} são projetados com a intenção de facilitar o desenvolvimento de \textit{software}, habilitando designers e programadores a gastarem mais tempo determinando as exigências do \textit{software} do que com detalhes de baixo nível do sistema}

\newword{17}{Dojo Toolkit}{é uma biblioteca em JavaScript, de código fonte aberto, projetado para facilitar o rápido desenvolvimento de interfaces ricas}

\newword{18}{Meta-tag}{Uma etiqueta HTML identificando o conteúdo de um \textit{website}. Informações comumente encontradas na meta-tag incluem: direitos autorais, palavras-chave para ferramentas de busca e descrições da formatação da página}

\newword{19}{DOCTYPE}{\textit{Document Type Definition} ou Definição do tipo de documento. Indica para o navegador e para outros meios qual a especificação de código utilizar. O DOCTYPE não é uma \textit{tag} do HTML, mas uma instrução para que o navegador tenha informações sobre qual versão de código a marcação foi escrita}

\newword{20}{Template}{é um documento sem conteúdo, com apenas a apresentação visual (apenas cabeçalhos por exemplo) e instruções sobre onde e qual tipo de conteúdo deve entrar a cada parcela da apresentação}

\newword{21}{Padrões de projeto}{ou \textit{Design Pattern}, descreve uma solução geral reutilizável para um problema recorrente no desenvolvimento de sistemas de \textit{software} orientados a objetos. Não é um código final, é uma descrição ou modelo de como resolver o problema do qual trata, que pode ser usada em muitas situações diferentes}

\newword{22}{e-MAG}{Modelo de Acessibilidade de Governo Eletrônico. Consiste em um conjunto de recomendações a ser considerado para que o processo de acessibilidade dos sítios e portais do governo brasileiro seja conduzido de forma padronizada e de fácil implementação}

\newword{23}{Web}{Sinônimo mais conhecido de \textit{World Wide Web} (WWW). É a interface gráfica da Internet que torna os serviços disponíveis totalmente transparentes para o usuário e ainda possibilita a manipulação multimídia da informação}

% Comando para incluir todas as definições do arquivo glossario.tex
%\glsaddall
% Impressão do glossário
%\printglossaries

% ----------------------------------------------------------
% Apêndices
% ----------------------------------------------------------

% ---
% Inicia os apêndices
% ---
\begin{apendicesenv}

%\chapter{Documento Básico Usando a Classe icmc}
%\label{chapter:documento-basico}
%
\begin{codigo}[caption={Exemplo de um documento básico}, label={codigo:documento-basico}, language=Tex, breaklines=true]
% Documento utilizando a classe ufgcac
% Opções: 
%   Qualificação         = qualificacao 
%   Curso                = doutorado/mestrado
%   Situação do trabalho = pre-defesa/pos-defesa (exceto para qualificação)
% -- opções do pacote babel --
% Idioma padrão = brazil
	%french,	    % idioma adicional para hifenização
	%spanish,			% idioma adicional para hifenização
	%english,			% idioma adicional para hifenização
	%brazil				% o último idioma é o principal do documento
\ documentclass[doutorado, spanish, english, brazil]{packages/icmc}

% Título do trabalho
\titulo{Título da Monografia}

% Nome do autor
\autor[Abreviação]{Nome completo do autor}

% Define o local
\local{São Carlos -- SP}

% Data do depósito
\data{18}{12}{2012}

% Nome do Orientador
\orientador[Orientador:]{Titulação do orientador}{Nome completo do Orientador}

% Nome do Coorientador (caso não exista basta remover)
\coorientador[Coorientador:]{Titulação do coorientador}{Nome completo do Coorientador}
% Se coorientadora troque Coorientador: por Coorientadora: dentro do colchetes

% Define o nome da instituição
\instituicao{Instituto de Ciências Matemáticas e de Computação (ICMC/USP)}

% Especiolidade e Nome do programa de Pós-graduação
\curso[Ciências -- Ciências de Computação e Matemática Computacional]{Ciências de Computação e Matemática Computacional}
% O valor entre colchetes é opcional

% Resumo
\textoresumo[Idioma]{
Texto do resumo do trabalho.
}{Lista de palavras-chave separada por virgulas}


% Início do documento
\begin{document}

\chapter{Introdução}

Capítulo de Introdução

\chapter{Desenvolvimento}

Capítulo de Desenvolvimento

\chapter{Conclusão}

Capítulo de conclusão

% Nome do arquivo com as referências bibliográficas
\bibliography{referencias}

\end{document}

\end{codigo}

\end{apendicesenv}
% ---


% ----------------------------------------------------------
% Anexos
% ----------------------------------------------------------

% ---
% Inicia os anexos
% ---
\begin{anexosenv}

%\chapter{Páginas Interessantes na Internet} 
%\label{chapter:paginas-interessantes}
%\begin{description}
 \item[\url{http://www.tex-br.org}]: Página em português com diversos tutoriais e referências interessantes sobre \LaTeX;
 \item[\url{http://en.wikibooks.org/wiki/LaTeX}]: Livro em formato \textit{wiki} gratuito sobre \LaTeX;
 \item[\url{http://tobi.oetiker.ch/lshort/lshort.pdf}]: Ótimo tutorial sobre \LaTeX (possui versão em português \url{http://alfarrabio.di.uminho.pt/~albie/lshort/ptlshort.pdf}, mas a versão em inglês é a mais atual);
 \item[\url{http://code.google.com/p/abntex2/}]: Página do abnTeX2, grupo que desenvolve os pacotes e classes em \LaTeX para as normas da ABNT, nos quais a classe \textit{icmc} foi baseada;
\item[\url{ http://www.rexlab.ufsc.br:8080/more/index.jsp}]: Página do Mecanismo On-line para Referências  (MORE) desenvolvido pela UFSC.
 \end{description}

\end{anexosenv}
% ---

\end{document}