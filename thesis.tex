% ------------------------------------------------------------------------
% ------------------------------------------------------------------------
% ICMC: Modelo de Trabalho Acadêmico (tese de doutorado, dissertação de
% mestrado e trabalhos monográficos em geral) em conformidade com 
% ABNT NBR 14724:2011: Informação e documentação - Trabalhos acadêmicos -
% Apresentação
% ------------------------------------------------------------------------
% ------------------------------------------------------------------------

% Opções: 
%   Qualificação         = qualificacao 
%   Curso                = doutorado/mestrado
%   Situação do trabalho = pre-defesa/pos-defesa (exceto para qualificação)
% -- opções do pacote babel --
% Idioma padrão = brazil
	%french,	    	% idioma adicional para hifenização
	%spanish,			% idioma adicional para hifenização
	%english,			% idioma adicional para hifenização
	%brazil				% o último idioma é o principal do documento
\documentclass[brazil]{packages/icmc}

% ---
% Pacotes Opcionais
% ---
\usepackage{rotating}           % Usado para rotacionar o texto
\usepackage[all,knot,arc,import,poly]{xy}   % Pacote para desenhos gráficos
% Este pacote pode conflitar com outros pacotes gráficos como o ``pictex''
% Então é necessário usar apenas um dos pacotes conflitantes


% ---
% Informações de dados para CAPA e FOLHA DE ROSTO
% ---
\titulo{Estudo de ataques DoS em redes MQTT}
\autor[Cruz, G. M.]{Gabriel de Melo Cruz}
\orientador[Orientadora:]{Profa. Dra.}{Kalinka Regina Lucas Jaquie Castelo Branco}
%\coorientador[Coorientadora:]{Doutoranda}{Mariana Rodrigues}
\curso{Computação}
\area{Segurança em Redes de Computadores} % Área de concentração do trabalho
\data{12}{06}{2019} % Data do depósito
% ---

% ---
% RESUMOS
% ---

% Resumo em português
% conter no máximo 500 palavras
\textoresumo{
    O desenvolvimento de tecnologias associadas à Internet das Coisas está hoje mais rápido do que nunca. Diante desse cenário de evolução desenfreada, simples práticas de segurança frequentemente não são contempladas, deixando esses sistemas expostos através de várias vulnerabilidades. Neste trabalho exploramos falhas que permitem desabilitar o serviço em um dos protocolos mais utilizados por sistemas IoT: o MQTT. A primeira vulnerabilidade explorada leva em conta a utilização do protocolo TCP -- que oferece ao MQTT o serviço de conexão e entrega de mensagens -- incapacitando a criação de novas conexões. Ainda, exploramos o limite de banda do servidor, ou broker, no protocolo MQTT, fazendo com que ele ocupe-a unicamente com as requisições do atacante. Os dois ataques aqui implementados, apesar de relativamente simples, são extremamente eficazes sobretudo devido à falta de implementações de software seguras em sistemas IoT.
    }{IoT, Internet das Coisas, MQTT, Segurança, Ataque, Vulnerabilidade}

% ---
% resumo em inglês
% ---
\textoresumo[english]{
    The development of technologies associated with the Internet of Things is faster than ever before. Facing this scenario of unbridled evolution, simple security practices are often not contemplated, leaving these systems exposed through numerous vulnerabilities. In this work we explore failures that allow us disable the service in one of the most used protocols on IoT systems: MQTT. The first exploited vulnerability makes use of a problem in the TCP protocol -- which provides MQTT with the connection and delivery of messages --, allowing us to prevent new connections on the server, or broker. We also exploit issues regarding the bandwidth limit of the broker, occupying it exclusively with the attacker's requests. Even though the two attacks implemented here are relatively simple, they are also extremely effective, mainly because of the lack of secure software implementations in IoT systems.
    }{IoT, Internet of Things, MQTT, Security, Attack, Vulnerability}
    
% ---
% Configurações de aparência do PDF final
% ---
% alterando o aspecto da cor azul
\definecolor{blue}{RGB}{41,5,195}

% informações do PDF
\makeatletter
\hypersetup{
     	pagebackref=true,
		pdftitle={\@title}, 
		pdfauthor={\@author},
    	pdfsubject={\imprimirpreambulo},
	    pdfcreator={LaTeX with abnTeX2/ICMC-USP},
		pdfkeywords={\palavraschave}, 
		colorlinks=true,       		% false: boxed links; true: colored links
    	linkcolor=blue,          	% color of internal links
    	citecolor=blue,        		% color of links to bibliography
    	filecolor=magenta,      	% color of file links
		urlcolor=blue,
		bookmarksdepth=4
}
\makeatother
% --- 

% ----------------------------------------------------------
% ELEMENTOS PRÉ-TEXTUAIS
% ----------------------------------------------------------

% Inserir a ficha catalográfica
%\incluifichacatalografica[tex/fichaCatalografica.pdf]
%\incluifichacatalografica


% Inserir folha de aprovação
%\input{tex/folha-aprovacao}

% DEDICATÓRIA / AGRADECIMENTO / EPÍGRAFE
\textodedicatoria*{tex/pre-textual/dedicatoria}
% \textoagradecimentos*{tex/pre-textual/agradecimentos}
\textoepigrafe*{tex/pre-textual/epigrafe}

% Inclui a lista de figuras
\incluilistadefiguras

% Inclui a lista de tabelas
%\incluilistadetabelas

% Inclui a lista de quadros
%\incluilistadequadros

% Inclui a lista de algoritmos
%\incluilistadealgoritmos

% Inclui a lista de códigos
\incluilistadecodigos

% Inclui a lista de siglas e abreviaturas
\incluilistadesiglas

% Inclui a lista de símbolos
%\incluilistadesimbolos

% ----
% Início do documento
% ----
\begin{document}

% ----------------------------------------------------------
% ELEMENTOS TEXTUAIS
% ----------------------------------------------------------
\textual

\chapter{Introdução}
\label{chapter:introducao}
\section{Motivação e Contextualização}
A Internet das Coisas (em inglês, Internet of Things ou IoT) é um conceito em grande expansão atualmente (TODO: REFERÊNCIA). As máquinas de IoT são em sua grande maioria sensores, microcontroladores e sistemas embarcados \-- em suma, sistemas computacionais pequenos, que gastam pouca energia e possuem baixo poder computacional. 

A popularização da Internet das Coisas ocorreu tendo como principais focos o baixo consumo energético, leveza computacional (TODO: LEVEZA?), e solução de diversos problemas de viabilidade em sistemas críticos como, por exemplo, máquinas que precisam operar em ambientes de altas temperatura e pressão (TODO: REFERÊNCIA).

Esse desenvolvimento acelerado, no entanto, deixou de lado um aspecto importante de qualquer sistema computacional: segurança. Logo, percebeu-se uma gama de falhas de segurança em divesos protocolos de comunicação (TODO: REFERÊNCIA) projetados para sistemas IoT, bem como a falta de preocupação com criptografia e outras práticas de segurança.

Essa crise de segurança despertou uma enorme necessidade de protocolos seguros, como aqueles mais tradicionais (TODO: REFERÊNCIA) \-- utilizados há anos fora da Internet das Coisas \-- mas que ainda possuíssem as propriedades de baixos custos energético e computacional, intrínsecas à IoT.


\subsection{Modelo Publish/Subscribe}
A Internet das Coisas, como o próprio nome dá a entender, faz parte da Internet, ou seja, utiliza os protocolos TCP/IP para comunicação (TODO: REFERÊNCIA), no entanto, protocolos como HTTP e HTTPS, largamente utilizado nas demais camadas da rede, não são adequados para a maioria das aplicações IoT devido a restrições como baixa largura de banda e alta latência (TODO: REFERÊNCIA). O modelo Publish/Subscribe é um dos modelos de comunicação mais utilizados por dispositivos IoT, sobretudo em redes de sensores (TODO: REFERÊNCIA).

Nesse modelo os clientes, máquinas que se comunicam, se conectam a um corretor, ou broker em inglês, e se inscrevem (TODO: subscribe) em tópicos. Os clientes podem ainda publicar (TODO: publish) mensagens em um ou mais tópicos, e todos os clientes que estiverem inscritos nesses tópicos receberão as mensagens.

TODO: IBAGENS!

Esse modelo permite que apenas o broker despenda uma quantidade considerável de recursos energéticos e computacionais \-- para armazenar as listas dos clientes inscritos, garantir níveis de qualidade de serviço, dentre outras tarefas, \-- enquanto os clientes somente enviam e recebem as mensagens, não precisando nem mesmo checar os tópicos ativamente (TODO: MELHORAR, REFERÊNCIA, BUSY WAIT).


\subsection{MQTT}
O MQTT, Message Queuing Telemetry Transfer, é um protocolo publish/subscribe criado em 1999 por Andy Stanford-Clark e Arlen Nipper para [], e é o que usaremos para estudo de caso

-qos
-achou mercado agora com iot


\subsection{DoS}
- sigla
- histórico
- iot
- mqtt (publish/subscribe



\section{Objetivos}



\section{Organização}



% Comando simples para exibir comandos Latex no texto
\newcommand{\comando}[1]{\textbf{$\backslash$#1}}

Este documento explica brevemente como trabalhar com a classe \LaTeX~\textit{icmc} para confeccionar trabalhos acadêmicos seguindo as normas da \sigla{ABNT}{Associação Brasileira de Normas Técnicas} e as \aspas{\textit{Diretrizes para apresentação de dissertações e teses da USP: documento eletrônico e impresso. Parte I (ABNT)}}, publicado pelo \sigla{SIBi}{Sistema Integrado de Bibliotecas} USP. O presente manual também atende as exigências prevista no regimento do Programa de Pós-graduação em \sigla{CCMC}{Ciências da Computação e Matemática Computacional} do \sigla{ICMC}{Instituto de Ciências Matemáticas e de Computação} da \sigla{USP}{Universidade de São Paulo}.


A classe \textit{icmc} foi construída com base na última versão da classe \textit{abntex2} e do pacote \textit{abntex2cite}. Portanto, este documento exemplifica a elaboração de trabalho
acadêmico (tese, dissertação e outros do gênero) produzido conforme a ABNT NBR
14724:2011 \textit{Informação e documentação - Trabalhos acadêmicos - Apresentação}.

Assim, é altamente recomendável que seja consultada a documentação do \textit{abntex2}\footnote{https://code.google.com/p/abntex2/downloads/list}. A classe \textit{abntex2} foi desenvolvida para facilitar a escrita de documentos seguindo as normas da ABNT no ambiente \LaTeX\;\cite{frasson:2005:classe_abnt}.

Todo o trabalho de pesquisa e ajustes da presente classe \LaTeX~\emph{icmc} foram feitos pelo aluno mestrado do Programa de Pós-graduação em Ciência da Computação e Matemática Computacional, Humberto Lidio Antonelli, durante a confecção da sua monografia de qualificação.

O requisito básico para utilização da classe \textit{icmc} é criar um documento desta classe com o comando
\comando{documentclass[@parameters]\{icmc\}} e ter, no diretório de trabalho, o arquivo \emph{icmc.cls} presente. Entretanto, recomenda-se fortemente manter a estrutura de diretório inicial fornecida por este modelo.

Os parâmetros possíveis utilizados pelo \comando{documentclass} são:
\begin{description}
\item[french, spanish, english, brazil] Adiciona o idioma para correta hifenização correta no documento. O último idioma declarado é o principal do documento. O valor padrão é \textbf{brazil}.
\end{description}



\chapter{Métodos, Técnicas e Tecnologias Utilizadas}
\label{chapter:metodos}

\section{Message Queuing Telemetry Transport}

Criado em 1999 por Andy Stanford-Clark e Arlen Nipper, o \sigla{Message Queuing Telemetry Transfer}{MQTT} é um protocolo que utiliza o modelo publish/subscribe e tem como principal objetivo a baixa utilização de banda da rede. O MQTT encontrou um mercado promissor ainda recentemente, com a popularização da Internet das Coisas.

Os agentes do protocolo MQTT são o corretor, ou \textit{broker} em inglês, e os clientes. O broker é responsável por receber e enviar publicações, manter listas dos clientes inscritos em cada tópico, dentre outras atividades. Os clientes, por outro lado, possuem pouquíssimas responsabilidades nesse modelo --- eles devem apenas esperar pelas mensagens publicadas nos tópicos em que estão inscritos e, em alguns casos, mandar pacotes de confirmação de recebimento de mensagens do broker \cite{OasisMQTT}.

No MQTT, a existência de um broker concentrador de tarefas permite que os clientes despendam pouquíssimos recursos.

\section{Denial of Service}
Ataques de negação de serviço, ou \sigla{DoS}{\textit{Denial of Service}}, buscam desestabilizar um sistema de forma que clientes legítimos não consigam acessá-lo. Ataques DoS frequentemente utilizam ferramentas que realizam uma série requisições repetidas para o servidor alvo com vistas a ocupar canais de clientes legítimos e, assim, comprometer a disponibilidade do sistema \cite{Douligeris2003}.

Apesar desse tipo de vulnerabilidade já ter sido explorada e estudada há anos, recentemente, por conta da popularização da Internet das Coisas, novos problemas relacionados a ela surgiram, sobretudo no que tange a IoT.


\section{Mosquitto}
O protocolo MQTT, assim como qualquer protocolo em redes de computadores, não é senão um conjunto de especificações padronizadas para comunicação de máquinas. Sendo assim, existem várias diferentes implementações do protocolo MQTT, cada qual com suas particularidades em relação a modos de armazenamento, algoritmos de decisão etc.

O Mosquitto \cite{Mosquitto}, projeto desenvolvido pela Eclipse Foundation \cite{Eclipse}, visa a implementar os componentes especificados no padrão do protocolo MQTT. Dentre eles, destacam-se o corretor, Mosquitto broker, e o cliente, Paho.

O Mosquitto permite uma fácil utilização, basta baixá-lo e, em seguida, executar o seguinte comando (em máquinas Linux):

\begin{lstlisting}[language=bash, caption=Execução Mosquitto broker]
    $ mosquitto
\end{lstlisting}

Isso iniciará o broker, ouvindo na porta TCP 1883. É possível também especificar a porta através da opção \textbf{-p}. A listagem a seguir demonstra a execução do broker na porta 5050:

\begin{lstlisting}[language=bash, caption=Execução Mosquitto broker na porta 5050]
    $ mosquitto -p 5050
\end{lstlisting}




\section{Paho}

O Paho é a implementação do cliente MQTT do projeto Mosquitto, também mantido pela Eclipse Foundation \cite{Paho}. A função do Paho é a de prover uma interface simples para a implementação de clientes MQTT, contando com funções que permitem conexão com o broker, inscrição em tópicos, publicações, dentre outras funcionalidades.

O Paho é um projeto que conta com implementações equivalentes em uma série de diferentes linguagens de programação, como Python, Java, C/C++, JavaScript etc. Neste trabalho, usaremos a biblioteca Python do Paho, uma vez que os códigos aqui desenvolvidos utilizam essa mesma linguagem.


\chapter{Desenvolvimento}
\label{chapter:desenvolvimento}

\section{SYN Flood}

O ataque de \emph{SYN flood} já é bem conhecido e explora uma vulnerabilidade tradicional no processo de \emph{handshaking} do protocolo TCP, com vistas a impedir que o alvo faça conexões legítimas e causando, assim, negação de serviço \cite{Yuan2008}.


\subsection{TCP three-way handshake}
O \emph{three-way handshake} do protocolo TCP é um processo que ocorre durante a fase de conexão entre duas máquinas, garantindo que ambas estão cientes da conexão e prontas para trocar mensagens. (TODO: REF)

Nesse processo, ilustrado pela \autoref{three-way-handshake}, há a troca de três mensagens entre cliente e servidor -- chamaremos de cliente a máquina que inicia a conexão e servidor a que aceita novas conexões:

\begin{enumerate}
  \item \textbf{SYN}: Um pacote de sincronização é enviado do cliente para o servidor, sinalizando sua intenção de se conectar.
  \item \textbf{SYN/ACK}: O servidor responde, reconhecendo a conexão.
  \item \textbf{ACK}: O cliente avisa o servidor de que recebeu o pacote SYN/ACK e a conexção é estabelecida.
\end{enumerate}




\begin{figure}[htb]
 \caption{Three-way handshake}
 \label{three-way-handshake}
 \centering
 \includegraphics[scale=0.6]{images/three-way-handshake.png}
 \fautor
\end{figure}




\subsection{Vulnerabilidade}

O SYN flood consiste em mandar uma série de pacotes SYN para um servidor e, ao receber os respectivos pacotes SYN/ACK, simplesmente ignorá-los. Dessa maneira, o servidor fica esperando pela resposta ACK que nunca chegará. Eventualmente o tempo limite de espera do servidor para cada conexão expira mas, logo em seguida, o atacante inicia novas conexões, o que acaba por manter o servidor incapaz de receber conexões legítimas -- caracterizando a negação de serviço do servidor. (TODO: REFERÊNCIAS)


\subsection{Ataque ao Mosquitto Broker}
O SYN flood é um ataque que explora uma vulnerabilidade do protocolo TCP, não no MQTT. No entanto, como o protocolo MQTT depende do TCP para realizar conexões e trocar mensagens, toda e qualquer vulnerabilidade encontrada no protocolo TCP pode também ser utilizada para atacar sistemas baseados em MQTT.

O ambiente utilizado para realizar o ataque foi:
\begin{enumerate}
    \item Mosquitto MQTT Broker (TODO: LINK/REFERÊNCIA) em uma máquina Linux
    
    \item Máquina atacante, também Linux, na mesma rede local, \sigla{LAN}{\textit{Local Area Network}} que o broker
    
    \item hping3 (TODO: LINK/REFERÊNCIA): ferramenta desenvolvida pela Offensive Security (TODO: LINK) para a criação, envio e análise de pacotes
\end{enumerate}

O hping3 foi escolhido em detrimento de outras implementações diretas do SYN flood uma vez que se trada de uma ferramenta frequentemente usada nos meios acadêmico e profissional, sendo inclusive de fácil uso e grande versatilidade. 




\begin{lstlisting}[language=bash, caption=SYN Flood]
# hping3 -c 15000 -d 120 -S -p 1883 --flood 192.168.1.159
\end{lstlisting}


Os parâmetros especificados são:

\begin{enumerate}
    \item \textbf{-c}: Número de pacotes a serem enviados.
    \item \textbf{-d}: Tamanho, em bytes, de cada pacote.
    \item \textbf{-S}: Especifica que os pacotes contém a flag SYN, ou seja, que são pacotes SYN.
    \item \textbf{-p}: Porta a receber o ataque. O servidor não conseguirá receber outras conexões nessa porta. A porta 1883 é a padrão para o protocolo MQTT.
    \item \textbf{--flood}: Especifica que trata-se de um ataque de \emph{flooding}, hping3 enviará os pacotes o mais rápido possível. (TODO: VERIFICAR ISSO)
    \item \textbf{192.168.0.2}: Endereço IP do servidor. (máquina alvo)
\end{enumerate}






\subsection{SYN flood e IP spoofing}

O SYN flood é um ataque facilmente mitigável. Uma proposta amplamente aceita e de fácil implementação é a limitação do número de conexões permitidas para um mesmo endereço IP. Assim, após determinado número de conexões abertas com um cliente em um determinado endereço IP, todas as outras tentativas de conexão advindas desse mesmo endereço seriam automaticamente rejeitadas pelo servidor.

Como essa solução se baseia principalmente no conhecimento do endereço IP do cliente pelo servidor, é possível atrelar ao ataque uma técnica de mascaramento de IP, ou \emph{IP spoofing}, de forma que o servidor seja incapaz de diferenciar conexões maliciosas de conexões legítimas, tornando essa solução ineficaz.



Simulamos também um ataque SYN flood com IP spoofing:

\begin{lstlisting}[language=bash, caption={SYN Flood com IP spoofing}, label={lst:hping}]
# hping3 -c 15000 -d 120 -S -p 1883 --flood --rand-source 192.168.1.159
\end{lstlisting}

A opção \textbf{--rand-source} permite aleatorizar o IP de origem especificado nos pacotes enviados pelo cliente.
(TODO: CÓDIGO?, O QUE MAIS?)












\section{Sybil subscription flood}

Como o protocolo MQTT opera de acordo com o modelo publish/subscribe é também possível realizar ataques que exploram vulnerabilidades atreladas não a características específicas desse protocolo, mas do modelo publish/subscribe especificamente.




\subsection{Ataque Sybil}

O ataque Sybil consiste fundamentalmente em simular uma grande quantidade de identidades falsas, fazendo com que um cliente sybil, ou nó sybil, no contexto de redes ponto a ponto, se passe por vários clientes. Isso pode ser atingido usando uma série de diferentes técnicas como por exemplo a criação de identidades falsas ou o roubo de identidades verdadeiras fazendo com que a autoridade -- frequentemente um servidor, ou, no caso da topologia no protocolo MQTT, o broker -- acredite que as identidades roubadas pelo nó sybil são os clientes legítimos \cite{Douceur2002}.

Esse tipo de ataque se mostra muito efetivo no contexto de redes MQTT, uma vez que na grande maioria das implementações o broker responsável não limita o número de clientes que podem operar sob um mesmo endereço IP.








\subsection{Vulnerabilidade}

A grande maioria dos brokers MQTT, dentre eles o Mosquitto Broker, permite por padrão um número ilimitado de inscrições em qualquer tópico mas, apesar disso, os brokers geralmente não possuem os aparatos necessários para lidar com um tráfego muito grande -- problemas frequentes são falta de banda e baixo poder de processamento.

Sendo assim, é possível realizar um ataque que explora essa inconsistência entre o número de inscrições permitidas e o número de conexões com que o broker consegue de fato lidar. Para isso, faz-se uma série de inscrições em qualquer tópico de um broker e, em seguida, realiza-se uma série de publicações nesse mesmo tópico. Como o tópico usado no ataque possui muitas inscrições, uma publicação demorará muito até que ela seja completada, ou seja, até que o broker seja capaz de enviar a mensagem para todos os clientes inscritos.

Teoricamente, esse ataque explora tanto a capacidade da rede na qual o broker está conectado quanto sua capacidade de processar requisições e enviar respostas. Contudo, na prática, apenas o primeiro fator acaba por ser relevante (TODO: DADOS?) e, portanto, trata-se em efeito de um ataque que explora a largura de banda do broker.






\subsection{Ataque ao Mosquitto Broker}

Para explorar a vulnerabilidade foi desenvolvido um programa na linguagem Python que utiliza a biblioteca Paho do projeto Mosquitto para criar e instanciar clientes, abrir conexões e realizar inscrições e publicações.

\begin{lstlisting}[language=Python, caption=Ataque Sybil]
    import paho.mqtt.client as mqtt
    import random, string, sys
    
    
    def gen_cli_id(size):
        return ''.join([random.choice(string.ascii_letters +  \                     string.digits) for n in range(size)])
    
    
    if __name__ == '__main__':
        if len(sys.argv) != 5:
            print('usage: python3 bad_sub.py HOST \
                  PORT TOPIC SYBIL_NODES')
            exit(0)
    
        host = sys.argv[1]
        port = int(sys.argv[2])
        topic = sys.argv[3]
        sybil_nodes = int(sys.argv[4])
        clients = []
    
        print('Subscribing sybil nodes...')
        for i in range(sybil_nodes):
            clients.append(mqtt.Client(gen_cli_id(10)))
            clients[i].connect(host, port)
            clients[i].subscribe(topic)
    
        print('Publishing messages, flooding broker...')
        i = 0
        while True:
            clients[i].publish(topic, 'flooding topic')
            i = (i + 1) % sybil_nodes

\end{lstlisting}

Nesse código, primeiro cria-se uma lista de vários clientes, cada qual com um ID aleatório gerado automaticamente. Em seguida, inscreve-se todos os clientes no tópico especificado e, finalmente, são enviadas várias publicações nesse mesmo tópico.

O programa gera uma utilidade de linha de comando (\emph{command line utility}) que permite escolher parâmetros como IP (HOST) e porta (PORT) da máquina onde o broker está rodando, bem como tópico a ser atacado (TOPIC) e número de clientes falsos a serem gerados (SYBIL\_NODES).

\begin{lstlisting}[language=Bash, caption={Utilização da ferramenta}, label={lst:sybil}]
    $ python3 bad_sub.py
    usage: python3 bad_sub.py HOST PORT TOPIC SYBIL_NODES
\end{lstlisting}








\subsection{Estratégias de mitigação e sofisticações do ataque}

Esse ataque, da maneira como foi implementado, é passível de algumas estratégias de mitigação relativamente simples. No entanto, como veremos, cada qual também possui vulnerabilidades que modificações igualmente simples do ataque são capazes de explorar.

\subsubsection{IP blocking}

A solução imediata, e talvez até ingênua, é bloquear qualquer requisição advinda do endereço IP da máquina maliciosa após a detecção do ataque, seja ela publicação ou inscrição.

Essa abordagem não compõe uma estratégia de mitigação \emph{a priori}, uma vez que é eficaz apenas depois da detecção do ataque. No entanto, como será visto na próxima seção, a ideia do bloqueio de endereços IP é útil para se conceber uma estratégia razoavelmente eficaz para evitar esse tipo de ataque.

Muito embora essa solução seja eficaz contra o ataque apresentado anteriormente, o \emph{IP blocking} pode ser facilmente superado utilizando a técnica de mascaramento de IP, ou \emph{IP spoofing}, em que o atacante altera o endereço de origem especificado nos pacotes IP por ele enviados, fazendo com que o broker seja incapaz de detectar o verdadeiro endereço IP do atacante e, assim, igualmente incapaz de bloqueá-lo.



\subsubsection{Limitação de inscrições por IP}

Como visto anteriormente, bloquear determinados endereços IP não é uma estratégia capaz de evitar ataques, mas apenas de interromper ataques previamente detectados. Portanto, uma variação da técnica de IP blocking que promove maior robustez \emph{a priori} é estabelecer um limite máximo no número de inscrições por endereço IP. Assim, uma máquina de um determinado endereço IP não seria capaz de inscrever vários clientes maliciosos no broker.

Mais uma vez, no entanto, alterações que incorporem técnicas de IP spoofing são extremamente efetivas contra essa estratégia de mitigação, uma vez que impossibilitam a detecção do verdadeiro endereço IP de origem dos clientes maliciosos.



\subsubsection{Limitar frequência de inscrições em um tópico}

Finalmente, é possível limitar a quantidade de inscrições em um determinado tópico por uma unidade de tempo, com vistas a tornar o ataque inviável devido ao tempo necessário para realizar as todas as inscrições necessárias para desabilitar o serviço. Por exemplo, permitir no máximo 40 inscrições por minuto em um mesmo tópico faria com que um ataque  de 8000 inscrições levasse cerca de três horas para completar a etapa de inscrições -- dando uma maior margem de tempo para que os administradores do sistema ajam.

Novamente, é possível alterar o código do ataque para torná-lo viável mesmo com essa estratégia. Agora, basta fazer com que os clientes gerados se inscrevam em múltiplos tópicos e, em seguida, realizar publicações em todos esses tópicos ao invés de apenas um, como anteriormente.




\section{Resultados}

\subsection{SYN flood}

Para realizar o ataque de SYN flood vamos executar o comando da listagem \ref{lst:hping}




\subsection{Sybil subscription flood}











\chapter{Conclusão}
\label{chapter:conclusao}

\section{Resultados}



\section{Dificuldades e Limitações}



%\chapter{Citações e Referências}
%\label{chapter:citacoes}
%\input{tex/citacoes}


% ---
% Finaliza a parte no bookmark do PDF, para que se inicie o bookmark na raiz
% ---
\bookmarksetup{startatroot}% 
% ---

% ----------------------------------------------------------
% ELEMENTOS PÓS-TEXTUAIS
% ----------------------------------------------------------
\postextual

% ----------------------------------------------------------
% Referências bibliográficas
% ----------------------------------------------------------
\bibliography{references}

% ---------------------------------------------------------------------
% GLOSSÁRIO
% ---------------------------------------------------------------------

% Arquivo que contém as definições que vão aparecer no glossário
%\input{tex/glossario}
% Comando para incluir todas as definições do arquivo glossario.tex
%\glsaddall
% Impressão do glossário
%\printglossaries

% ----------------------------------------------------------
% Apêndices
% ----------------------------------------------------------

% ---
% Inicia os apêndices
% ---
\begin{apendicesenv}

%\chapter{Documento Básico Usando a Classe icmc}
%\label{chapter:documento-basico}
%\input{tex/appendix/documento-basico}

\end{apendicesenv}
% ---


% ----------------------------------------------------------
% Anexos
% ----------------------------------------------------------

% ---
% Inicia os anexos
% ---
\begin{anexosenv}

%\chapter{Páginas Interessantes na Internet} 
%\label{chapter:paginas-interessantes}
%\input{tex/annex/paginas-interessantes}

\end{anexosenv}
% ---

\end{document}